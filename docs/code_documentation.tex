\documentclass[11pt]{tufte-handout}
\usepackage[utf8]{inputenc}
\usepackage[english]{babel}
\usepackage{amsthm}

% Encoding
\usepackage[utf8]{inputenc}

% Graphics, figures and plots
\usepackage{graphicx}
\usepackage{float}
\usepackage{pgfplots}
\usepgfplotslibrary{fillbetween}

% Etc
\usepackage{authblk}
\usepackage{hyperref}
\usepackage{xifthen}
\usepackage{verbatim} % Multiline comments with \begin{comment} and \end{comment}
\usepackage[inline,shortlabels]{enumitem} % Control vertical spacing: \begin{itemize}[noitemsep]
%\setlist{nolistsep} % No vertical space before \begin{itemize}
\usepackage{csquotes} % for block quotes

% Bibliography
\usepackage{natbib}

% Math
\usepackage{amsmath}
\usepackage{amsthm}
\usepackage{amsfonts}
\usepackage{amssymb}
\usepackage{amscd}
\usepackage{eurosym}
\usepackage{mathrsfs}
\usepackage{mathtools}
\usepackage{dsfont}
\usepackage{blkarray, bigstrut}
\usepackage{bm} % defines a command \bm which makes its argument bold
\usepackage{bbm}
\usepackage{nicefrac}

% Pseudocode
\usepackage{algorithm}
\usepackage{algpseudocode}
%\allowdisplaybreaks
% Use chapter/section numbers with tufte-handout
\setcounter{secnumdepth}{2}
% Margin notes for math mode
\def\mathnote#1{%
    \tag*{\rlap{\hspace\marginparsep\smash{\parbox[t]{\marginparwidth}{%
    \footnotesize#1}}}}
}

%%%% Mathematical Environments
\theoremstyle{plain}
\newtheorem{theorem}{Theorem}[section]
\newtheorem*{theorem-anon}{Theorem}
\newtheorem{lemma}[theorem]{Lemma}
\newtheorem*{lemma-anon}{Lemma}
\newtheorem{corollary}[theorem]{Corollary}
\newtheorem{proposition}[theorem]{Proposition}
\newtheorem{conjecture}[theorem]{Conjecture}

\theoremstyle{definition}
\newtheorem{definition}[theorem]{Definition}
\newtheorem{example}[theorem]{Example}

\theoremstyle{remark}
\newtheorem{remark}[theorem]{Remark}
\newtheorem{note}[theorem]{Note}
\newtheorem{case}[theorem]{Case}

%%%%%%%%%%
% Operators
\DeclareMathOperator*{\argmax}{arg\,max}
\DeclareMathOperator*{\argmin}{arg\,min}
\DeclareMathOperator*{\Exp}{\mathbb{E}}
\DeclareMathOperator*{\Var}{\text{Var}}
\DeclareMathOperator*{\Cov}{\text{Cov}}

%%%%%%%%%%
% Paired -- \command* will automatically resize
\DeclarePairedDelimiter\paren{(}{)}           % (parentheses)
\DeclarePairedDelimiter\ang{\langle}{\rangle} % <angle brackets>
\DeclarePairedDelimiter\abs{\lvert}{\rvert}   % |absolute value|
\DeclarePairedDelimiter\norm{\lVert}{\rVert}  % ||norm||
\DeclarePairedDelimiter\bkt{[}{]}             % [brackets]
\DeclarePairedDelimiter\set{\{}{\}}           % {braces}
%
\DeclarePairedDelimiter{\ceil}{\lceil}{\rceil}
\DeclarePairedDelimiter{\floor}{\lfloor}{\rfloor}

%%%%%%%%%%
% Number sets
\newcommand{\nsNatural}{\mathbb{N}} % Try to avoid.  Ambiguity if 0 is included.
\newcommand{\nsInteger}{\mathbb{Z}}
\newcommand{\nsRational}{\mathbb{Q}}
\newcommand{\nsReal}{\mathbb{R}}
\newcommand{\nsImaginary}{\mathbb{I}}
\newcommand{\nsComlex}{\mathbb{C}}
\newcommand{\E}{\mathbb{E}}

%%%%%%%%%%%%%%%%%%%%%%%%%%%%%%%%%%%%%%%%%%%%%%%%%%%%%%%%%%%%%%%%%%%%%%%%%%%%%%%%%%%%%%%%%%%%%%%%%%%%%%%%%%%%%%%%%%%%%%%%%%%%%%%%%%%%%%%%%%%%%%%%%%%%%%%%%%%%%%%%%%%%%%
%%% Fields %%%
\title{Sequence Form Solver Documentation and Test Cases}
\author{Reca Sarfati}
\date{March 2018}

\begin{document}
\maketitle

%%% Section 1
\setcounter{section}{0}
\section{Code Explained}
The file \verb|sequence_form.py| contains the following main methods: \\ \begin{itemize}[-]
    \item \verb|decorate_sequences(bt)| \\
    This function consumes a tree, and performs a breadth-first search so as to decorate each node with the unique sequences of player 1 and player 2's movement decisions from the root to itself. A decision is encoded as the index of the child, enumerating from left to right. (For example, if a given node has two children, and the sequence elects to move to the leftmost child, a `0' is stored). The method then returns a list of nodes.
    
    \item \verb|extensive_to_strategic_form(game)| \\ As it sounds, this method intakes a decorated game, and populates the $A$ and $B$ matrices, corresponding to players' payoff  matrices.
    
    \item \verb|extensive_to_sequence_form(game)| \\ This method, similarly, intakes a decorated game, and returns the $A,B,E,F,e,f,$ matrices.
    
    \item \verb|solve_strategic_form(game)| \\ This method computes player 1's strategy for inputted strategic form matrices $A,B$.
    
    \item \verb|solve_strategic_form(game)| \\ This method computes player 1's strategy for inputted strategic form matrices $A,B$.
    
    \item \verb|solve_strategic_form(game)| \\ This method computes player 1's strategy for inputted sequence form matrices $A,B,E,F,e,f$.
\end{itemize} 

 The file \verb|sequence_form_test.py| contains various named tests, annotated in the ensuing pages.

\newpage
\setcounter{section}{1}
\section{Tests}
\subsection{Von Stengel Test}
In this test, we solve the same tree as in the paper:
    \footnote{Colored blocks indicate information sets, where 1 implies player A's turn, and 2 implies player B's.} 
\\[6pt]
\begin{center}
    \includegraphics[width=4cm]{von_stengel_test.png}
\end{center}
The associated payoff matrices for players A and B are as follows:\\[6pt]
\begin{tabular}{p {4cm} c c}
     {$\displaystyle
        A = \begin{blockarray}{c c c c}
            \varnothing & l & r \\
            \begin{block}{[c c c] c}
                &   &   & \varnothing \\
                &   &   & L \\
              3 &   &   & R \\
                & 2 & 5 & LS \\
                & 0 & 6 & LT \\
            \end{block}
            \end{blockarray}$} &
        {$B = \begin{blockarray}{c c c c}
                \varnothing & l & r \\
                \begin{block}{[c c c] c}
                    &   &   & \varnothing \\
                    &   &   & L \\
                  3 &   &   & R \\
                    & 2 & 6 & LS \\
                    & 3 & 1 & LT \\
                \end{block}
                \end{blockarray}
    $}
\end{tabular}

The associated constraint matrices are as follows: \\[12pt]

$ E = \begin{blockarray}{c c c c c}
            \varnothing & L & R & LS & LT \\
            \begin{block}{[c c c c c]}
                1 &    &   &   & \\
               -1 & 1  & 1 &   & \\
                  & -1 &   & 1 & 1 \\
            \end{block}
            \end{blockarray},\quad\quad
    e = \begin{blockarray}{[c]}
                1 \\
                0 \\
                0 \\
            \end{blockarray}
$
\newline
$
    F = \begin{blockarray}{c c c}
            \varnothing & l & r \\
            \begin{block}{[c c c]}
                1 &    &    \\
               -1 & 1  & 1 \\
            \end{block}
            \end{blockarray}, \quad\quad
    f = \begin{blockarray}{[c]}
                1 \\
                0 \\
            \end{blockarray}
$
\newpage
\subsection{Test 2: Three Branches, Unbalanced}
In this test, we solve the following zero-sum game:
\\[6pt]
\begin{center}
    \includegraphics[width=7cm]{test2.png}
\end{center}
The associated payoff matrices for players A and B are as follows:\\[6pt]
\begin{tabular}{p {4.5cm} c c}
     {$\displaystyle
        A = \begin{blockarray}{c c c c c}
            \varnothing & l & r & m\\
            \begin{block}{[c c c c] c}
                &   &   & & \varnothing \\
              1 &   &   & & L \\
                &   & 4 & & R \\
                & 2 &  &  & RS \\
                & 3 &  & & RT \\
                &  &  &5 & RU \\                
            \end{block}
            \end{blockarray}$} &
        {$B = \begin{blockarray}{c c c c c}
                \varnothing & l & r & m\\
                \begin{block}{[c c c c] c}
                    &   &    & & \varnothing \\
                 -1 &   &    & & L \\
                    &   & -4 & & R \\
                    & -2 &  &  & RS \\
                    & -3 &  & & RT \\
                    &  &  & -5& RU \\
                \end{block}
                \end{blockarray}
    $}
\end{tabular}

The associated constraint matrices are as follows: \\[12pt]

$ E = \begin{blockarray}{c c c c c c}
            \varnothing & L & R & RS & RT & RU\\
            \begin{block}{[c c c c c c]}
                1 &    &   &   & \\
               -1 & 1  & 1 &   & \\
                  &  &  -1 & 1 & 1 \\
                  &  &  -1 &  & & 1 \\
            \end{block}
            \end{blockarray},\quad\quad
    e = \begin{blockarray}{[c]}
                1 \\
                0 \\
                0 \\
            \end{blockarray}
$
\newline
$
    F = \begin{blockarray}{c c c c}
            \varnothing & l & r & m\\
            \begin{block}{[c c c c]}
                1 &    &   & \\
               -1 & 1  & 1 & 1 \\
            \end{block}
            \end{blockarray}, \quad\quad
    f = \begin{blockarray}{[c]}
                1 \\
                0 \\
            \end{blockarray}
$
\newpage
\subsection{Test 3: Complete Binary Tree, Depth 3}
In this test, we solve the following zero-sum game on a complete binary tree:
\\[6pt]
\begin{center}
    \includegraphics[width=7cm]{test3.png}
\end{center}
The associated payoff matrices for players A and B are as follows:\\[6pt]
\begin{tabular}{p {4.5cm} c c}
     {$\displaystyle
        A = \begin{blockarray}{c c c c}
            \varnothing & l & r\\
            \begin{block}{[c c c] c}
               &   &  & \varnothing \\
               &   &  & L \\
               &   &  & R \\
               & 1 & 3 & LS \\
               & 2 & 4 & LT \\
               & 5 & 7 & RS \\
               & 6 & 8 & RT \\
            \end{block}
            \end{blockarray}$} &
        {$B = \begin{blockarray}{c c c c}
                \varnothing & l & r\\
                \begin{block}{[c c c] c}
               &   &  & \varnothing \\
               &   &  & L \\
               &   &  & R \\
               & -1 & -3 & LS \\
               & -2 & -4 & LT \\
               & -5 & -7 & RS \\
               & -6 & -8 & RT \\
                \end{block}
                \end{blockarray}
    $}
\end{tabular}

The associated constraint matrices are as follows: \\[12pt]

$ E = \begin{blockarray}{c c c c c c c}
            \varnothing & L & R & LS & LT & RS & RT \\
            \begin{block}{[c c c c c c c]}
                1 &    &   &   &  &  &\\
               -1 & 1  & 1 &   &  &  &\\
                  & -1 &   & 1 & 1   & \\
                  &    &-1 &   &  & 1 & 1 \\
            \end{block}
            \end{blockarray},\quad\quad
    e = \begin{blockarray}{[c]}
                1 \\
                0 \\
                0 \\
                0 \\
            \end{blockarray}
$
\newline
$
    F = \begin{blockarray}{c c c}
            \varnothing & l & r\\
            \begin{block}{[c c c]}
                1 &    &  \\
               -1 & 1  & 1 \\
            \end{block}
            \end{blockarray}, \quad\quad
    f = \begin{blockarray}{[c]}
                1 \\
                0 \\
            \end{blockarray}
$
\newpage
\subsection{Test 4: Highly Uneven Tree, Depth 5}
In this test, we solve the following zero-sum game on a highly uneven tree:
\\[6pt]
\begin{center}
    \includegraphics[width=7cm]{test4.png}
\end{center}
The associated payoff matrices for players A and B are as follows:\\[6pt]
$
        A = \begin{blockarray}{c c c c c c c c c}
            \varnothing & l & r & o & m & lh & oj & oq\\
            \begin{block}{[c c c c c c c c] c}
               & & & & & & &  &\varnothing \\
               & &3& &7& & &  &L \\
             8 & & & & & & &  &R \\
               &1& & & && &  &LS \\
               &-1& && & & &  &LT \\
               & & & & & 2& &  &LU \\
               & & & & &  &4&  &LV \\
               && &6& & & &  &LW \\  
               & & & & & & & 5&LVD \\
               & & & & & & & -5&LVG \\
            \end{block}
            \end{blockarray}$
\\
$B = \begin{blockarray}{c c c c c c c c c}
            \varnothing & l & r & o & m & lh & oj & oq\\
            \begin{block}{[c c c c c c c c] c}
               & & & & & & &  &\varnothing \\
               & &-3& &-7& & &  &L \\
             -8 & & & & & & &  &R \\
               &-1& & & && &  &LS \\
               &1& && & & &  &LT \\
               & & & & & -2& &  &LU \\
               & & & & &  &-4&  &LV \\
               && &-6& & & &  &LW \\  
               & & & & & & & -5&LVD \\
               & & & & & & & 5&LVG \\
            \end{block}
            \end{blockarray}$

The associated constraint matrices are as follows:
$$ E = \begin{blockarray}{c c c c c c c c c c c}
            \varnothing & L & R & LS & LT & LU & LV & LW & LVD & LVG \\
            \begin{block}{[c c c c c c c c c c c]}
                1 &    &   &   &  &&  &\\
               -1 & 1  & 1 &  &  & & &\\
                  & -1 &   &  1 & 1 & 1   & \\
                  & -1 & & &   &  & 1 & 1 \\
                  &  & & &   &  & -1 & & 1& 1 \\
            \end{block}
            \end{blockarray},\quad
        e = \begin{blockarray}{[c]}
                1 \\
                0 \\
                0 \\
                0 \\
            \end{blockarray}
$$
\newline
$
    F = \begin{blockarray}{c c c c c c c c c}
            \varnothing & l & r & o & m & lh & oj & oq \\
            \begin{block}{[c c c c c c c c c]}
                1 &    &  \\
               -1 & 1  & 1 & 1 & 1 & \\
                & -1 & & & & 1 \\
                & & & -1 & & & 1 & 1 \\
            \end{block}
            \end{blockarray}, \quad\quad
    f = \begin{blockarray}{[c]}
                1 \\
                0 \\
                0 \\
                0 \\
            \end{blockarray}
$
\end{document}
